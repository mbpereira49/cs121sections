%!TEX program = xelatex

\documentclass[11pt]{article}
\usepackage{amsmath,amssymb,amsthm}
\usepackage{filecontents}
\usepackage{graphicx}
\usepackage{listings}
\usepackage{xcolor}
\usepackage{colortbl}

%% SOLUTIONS TOGGLE
\newif\ifsolutions
%%% Un-comment to generate solutions:
%\solutionstrue
%%%

\DeclareMathOperator*{\E}{\mathbb{E}}
\let\Pr\relax
\DeclareMathOperator*{\Pr}{\mathbb{P}}

\newcommand{\eps}{\varepsilon}
\newcommand{\inprod}[1]{\left\langle #1 \right\rangle}
\newcommand{\R}{\mathbb{R}}

\newcommand{\handout}[5]{
  \noindent
  \begin{center}
  \framebox{
    \vbox{
      \hbox to 5.78in { {\bf CS 121: Introduction to Theoretical Computer Science } \hfill #2 }
      \vspace{4mm}
      \hbox to 5.78in { {\Large \hfill #5  \hfill} }
      \vspace{2mm}
      \hbox to 5.78in { {\em #3 \hfill #4} }
    }
  }
  \end{center}
  \vspace*{4mm}
}

\newcommand{\lecture}[4]{\handout{#1}{#2}{#3}{#4}{Section #1}}

\newtheorem{theorem}{Theorem}
\newtheorem*{proposition}{Proposition}
\newtheorem{lemma}[theorem]{Lemma}
\newtheorem{corollary}[theorem]{Corollary}
\newtheorem{conjecture}[theorem]{Conjecture}
\newtheorem{postulate}[theorem]{Postulate}
\theoremstyle{definition}
\newtheorem{defn}[theorem]{Definition}
\newtheorem{example}[theorem]{Example}
\newtheorem{exercise}{Exercise}
\newtheorem*{solution}{Solution}

\theoremstyle{remark}
\newtheorem*{remark}{Remark}
\newtheorem*{notation}{Notation}
\newtheorem*{note}{Note}

% \DeclareUnicodeCharacter{1F346}{\eggplant}

\newcommand{\sub}{\operatorname{sub}}
\newcommand{\quot}{\operatorname{quot}}
\newcommand{\bw}{\bigwedge}
\newcommand{\Avs}{\operatorname{Av}^{\operatorname{sign}}}
\newcommand{\bad}{\operatorname{bad}}
\newcommand{\sign}{\operatorname{sign}}
\newcommand{\id}{\operatorname{id}}
\newcommand{\defeq}{\vcentcolon=}
\newcommand{\eqdef}{=\vcentcolon}
%We can even define a new command for \newcommand!
\newcommand{\nc}{\newcommand}
\nc{\on}{\operatorname}
\nc\renc{\renewcommand}
\nc{\BR}{\mathbb R}
\nc{\BG}{\mathbb G}
\nc{\BP}{\mathbb P}
\nc{\BC}{\mathbb C}
\nc{\BQ}{\mathbb Q}
\nc{\BF}{\mathbb F}
\nc{\BZ}{\mathbb Z}
\nc{\BN}{\mathbb N}
\nc{\BS}{\mathbb S}
\nc{\Hom}{\on{Hom}}
\nc{\wt}{\widetilde}
\nc{\vspan}{\on{span}}
\nc{\ord}{\on{ord}}
\nc{\im}{\on{im}}
\nc{\Mat}{\on{Mat}}
\nc{\can}{\on{can}}
\nc{\coker}{\on{coker}}
\nc{\ev}{\on{ev}}
\nc{\Tr}{\on{Tr}}
\nc{\End}{\on{End}}
\nc{\swap}{\on{swap}}
\nc{\Set}{\on{Set}}
\nc{\bC}{{\mathbf C}}
\nc{\bc}{{\mathbf c}}
\nc{\bD}{{\mathbf D}}
\nc{\bd}{{\mathbf d}}
\nc{\bE}{{\mathbf E}}
\nc{\be}{{\mathbf e}}
\nc{\bF}{{\mathbf F}}
\nc{\bff}{{\mathbf f}}
\nc{\CE}{\mathcal E}
\nc{\CD}{\mathcal D}
\nc{\CH}{\mathcal H}
\nc{\CY}{\mathcal Y}
\renc{\mod}{\on{-mod}} %Careful - turn this off in a number theory setting
\newcommand{\spec}{\text{spec}}
\nc{\adj}{\on{adj}}
\nc{\tensor}[3]{#1 \underset{#2}\otimes #3}
\nc{\Nat}{\on{Nat}}
\nc{\op}{\on{op}}
\nc{\Funct}{\on{Funct}}
\nc{\Ob}{\on{Ob}}
\nc{\fR}{\mathfrak{R}}
\nc{\Vect}{\on{Vect}}
\nc{\ns}{\on{non-spec}}
\nc{\ol}{\overline}
\nc{\ul}{\underline}
\nc{\univ}{\on{univ}}
\nc{\Maps}{\on{Maps}}
\nc{\bdd}{\on{bdd}}
\nc{\cont}{\on{cont}}
\nc{\Sym}{\on{Sym}}
\nc{\vol}{\on{vol}}
\nc{\supp}{\on{supp}}
\nc{\Lie}{\on{Lie}}
\nc{\master}{\on{master}}
\nc{\pt}{\on{pt}}
% \nc{\dim}{\on{dim}}

\nc{\dy}{\on{dy}}

\providecommand{\tightlist}{%
  \setlength{\itemsep}{0pt}\setlength{\parskip}{0pt}}


% 1-inch margins, from fullpage.sty by H.Partl, Version 2, Dec. 15, 1988.
\topmargin 0pt
\advance \topmargin by -\headheight
\advance \topmargin by -\headsep
\textheight 8.9in
\oddsidemargin 0pt
\evensidemargin \oddsidemargin
\marginparwidth 0.5in
\textwidth 6.5in

\parindent 0in
\parskip 1.5ex

\usepackage{natbib}

\begin{document}
% Change these parameters accordingly!
\nc{\sectionnumber}{4}
\nc{\thedate}{}
\nc{\tfnames}{}


\lecture{\sectionnumber}{Fall 2019}{Prof.\ Boaz Barak}{\tfnames}


\section{Uncomputability}

Before we dive into the main topic of this section, we review the concept of computability.

\subsection{Recall: computability}

So far, the functions $F$ that we considered in this class had $\{0, 1\}^n$ as its domain, where
$n \in \mathbb{N}$. We've seen that $\emph{finite}$ functions

\begin{equation*}
    F:\{0, 1\}^n \rightarrow \{0, 1\}
\end{equation*}

are $\emph{computable}$ in the sense that we can always find a NAND-TM program $P_F$ such that
$P_F(s) = F(s)$ for all $s \in \{0, 1\}^n$. 

The question we ask is, $\emph{would this still be the case when the domain of the function is $\{0, 1\}^*$?}$
Recall that $\{0, 1\}^*$ is simply the set that contains binary strings of all lengths.
In other words, given any function $G$ with

\begin{equation*}
    G: \{0, 1\}^* \rightarrow \{0, 1\}
\end{equation*}

can we find a NAND-TM program $P_G$ with $P_G(s') = G(s')$ for all $s' \in \{0, 1 \}^*$?

As it turns out, we $\emph{can't}$ for some functions.
\bigskip

\subsection{Theorem: existence of an uncomputable function}

\begin{theorem}
    There exists a function that is $\emph{not}$ computable by any NAND-TM program.
\end{theorem}

$\emph{Intuition.}$
It is important to get the intuition here. There are infinitely many binary strings in $\{0, 1\}^*$, while there are
finite number of strings in $\{ 0, 1\}^n$. And that's precisely what makes it impossible to compute some functions since
NAND-TM programs are finite objects.

\subsection{Proof:}

\proof
Consider the set of all NAND-TM programs $P: \{0, 1\}^* \rightarrow \{0, 1\}$ (takes any binary input string and outputs one bit). 
Since $\emph{all}$ NAND-TM programs have an encoding, we can lexicographically order them
(they are countably infinite). Suppose that $(P_0, P_1, P_2, \dots)$ is the lexicographic ordering of all NAND-TM programs.

\begin{center}
    \begin{tabular}{c|cccccccc}
         & 0 & 1 & 10 & 11 & 100 & 101 & 110 & $\dots$ \\ 
        \hline
        $P_0$ & \cellcolor{green!20}0 & 1 & 1 & 0 & 1 & 1 & 1 & $\dots$ \\ 
        $P_1$ & 1 & \cellcolor{green!20}1 & 1 & 1 & 1 & 1 & 1 & $\dots$ \\ 
        $P_2$ & 1 & 1 & \cellcolor{green!20}0 & 1 & 1 & 1 & 1 & $\dots$ \\ 
        $P_3$ & 1 & 1 & 1 & \cellcolor{green!20}doesn't halt & 1 & 1 & 1 & $\dots$ \\ 
        $P_4$ & 1 & 1 & 1 & 1 & \cellcolor{green!20}1 & 1 & 1 & $\dots$ \\ 
        $P_5$ & 0 & 0 & 1 & 1 & 1 & \cellcolor{green!20}1 & doesn't halt & $\dots$ \\ 
        $P_6$ & 1 & 1 & 0 & 1 & 1 & 1 & \cellcolor{green!20}1 & $\dots$ \\ 
        $\vdots$ & $\vdots$ & $\vdots$ & $\vdots$ & $\vdots$ & $\vdots$ & $\vdots$ & $\vdots$ & 
    \end{tabular}
\end{center}

(Note: this table has been filled randomly just for the sake of illustrating the procedure.)

The first column of the above table, as we discussed, is just an ordering of $\emph{all}$ NAND-TM programs, and
the first row is the standard lexicogrphical ordering of all strings. Remember again that the first column contains
$\emph{ALL}$ NAND-TM programs. So $\emph{if we can construct a function that disagrees}$
$\emph{with all the programs}$ $\emph{(returns a different output for some
string}$
$\emph{from all of the programs in the
first column)}$, $\emph{that proves the claim}$.

Consider this function $F_{impossible}$ defined by flipping the bits in the green diagonal above.
Note that we just consider ``doesn't halt'' to be the same as 0.

\begin{center}
    \begin{tabular}{c|cccccccc}
        & 0 & 1 & 10 & 11 & 100 & 101 & 110 & $\dots$ \\ 
        \hline
        $F_{impossible}$ & \cellcolor{green!20}1 & \cellcolor{green!20}0 & \cellcolor{green!20}1 & \cellcolor{green!20}1
        &\cellcolor{green!20}0 & \cellcolor{green!20}0 & \cellcolor{green!20}0 & $\dots$
    \end{tabular}
\end{center}

Now the claim is that $F_{impossible}$ is different from all of the programs in the first column. $F_{impossible}$ is different from
the function simulated by
$P_0$, since they return different outputs for the string $0$. It is also different from $P_1$ since their outputs differ on $1$. 
Similarly, $P_2$ on $10$, and $P_3$ on $11$. It is not too difficult to see that $P_n$ is going to disagree with $F_{impossible}$ on the
binary representation of $n$.

Therefore, $F_{impossible}$ is different from all of the programs in the first column of the table, i.e.
no NAND-TM program can simulate $F_{impossible}$.

\subsection{Exercise}
Consider the set $P(\mathbb{N})$ of all subsets of $\mathbb{N}$. Show that there is no one-to-one and onto function between 
$\mathbb{N}$ and $P(\mathbb{N})$. 

\section{Reduction}
In the preceding section, we showed that there is some function from $\{0, 1\}^*$ to $\{0, 1\}$ that can't be simulated by
any NAND-TM program, i.e. an uncomputable function. However, the uncomputable function that we constructed seemed rather contrived.
After all, $F_{impossible}$ is constructed just so that it's different from all the NAND-TM programs in the list. In this section,
we look at the technique called $\emph{reduction}$ which can be used to show the uncomputability of some less contrived functions.

The big picture for reduction goes like this:
\begin{itemize}
    \item{
            You have a problem $A$ that you $\emph{know}$ you $\emph{can't}$ solve.
        }
    \item{
        And there's this other problem $B$ that you're wondering if you can solve.
        }
    \item{
            You imagine (assume) that $B$ is solvable ($\emph{this is for the sake of contradiction}$).
        }
    \item{
        As it turns out, if $B$ is solvable, then we can $\emph{use it}$ to solve $A$.
        }
    \item{
        But since $A$ is just simply not solvable, something that we assumed must've been wrong.
        }
    \item{
        So we deduce that $B$ can't be solvable (since that was the only assumption we made along the way).
        }
\end{itemize}

Using reduction, we now prove the following.

\subsection{Theorem: uncomputability of HALT}

\begin{theorem}
    Let $HALT: \{0, 1\}^* \rightarrow \{0, 1\}$ be the function such that
    \begin{equation*}
        HALT(P, x) = \begin{cases}
            0 & \text{$P$ halts on input $x$} \\
            1 & \text{otherwise}
        \end{cases}
    \end{equation*}

    Then $HALT$ is $\emph{not}$ computable.
\end{theorem}

The roadmap from above would look like the below in this particular case:

\begin{itemize}
    \item{
            We have a function $F_{impossible}$ that we know is not computable.
        }
    \item{
            And we are wondering if $HALT$ is computable.
        }
    \item{
            Assume for contradiction that $HALT$ is computable.
        }
    \item{
            If $HALT$ is computable, then $F_{impossible}$ should also be computable.
        }
    \item{
            But $F_{impossible}$ is not computable.
        }
    \item{
            Hence, $HALT$ couldn't have been computable.
        }
\end{itemize}

\subsection{Proof:}

\proof The idea is pretty clear from the roadmap above (hopefully?), so we just prove the crux of the argument.

\begin{center}
     If $HALT$ is computable, then $F_{impossible}$ should also be computable.
 \end{center}
 
Assume that $HALT$ is computable. Then there is some NAND-TM program $P_{haltsolver}$ that computes $HALT$. In other words,

   \begin{equation*}
       P_{haltsolver}(P, x) = \begin{cases}
            0 & \text{$P$ halts on input $x$} \\
            1 & \text{otherwise}
        \end{cases}
    \end{equation*}

    Using $P_{haltsolver}$ as a subroutine, we build $P_{impossiblesolver}$ as follows.

    Given input $s \in \{0, 1\}^*$, 
    \begin{enumerate}
        \item{
                Compute $n$, which is just the value of $s$ in decimal.
            }
        \item{
                Using $n$, it constructs $P_n$, which can be done in finite time (run down the
                lexicographically ordered list of all the strings until a valid description of $n^\text{th}$ NAND-TM
                program comes up).
            }
        \item{
                Run $P_{haltsolver}$ on $(P_n, s)$. If it tells us that $P_n$ halts on $s$, then we simply flip the output
                of $P_n$ on $s$ after it halts.

                If $P_{haltsolver}$ tells us that $P_n$ doesn't halt on $s$, return 1 (because we considered not halting to be
                the same as 0 earlier).
            }
    \end{enumerate}

    Now notice that the above builds exactly what we proved to be impossible in the previous theorem (i.e. $F_{impossible}$). Hence, something must've been wrong in the
    assumptions that we've made along the way, and we only made one assumption: $HALT$ is computable. We conclude that
    $HALT$ is $\emph{not}$ computable.

    
    \subsection{Exercise}
    Let $E$ be defined as follows.

   \begin{equation*}
       E(P) = \begin{cases}
           0 & \text{$P$ accepts any string from $\{0, 1\}^*$} \\
            1 & \text{otherwise}
        \end{cases}
    \end{equation*}

    Show that $E$ is $\emph{not}$ computable.


\section{Incompleteness}
\subsection{Introduction}
The notion of incompleteness explores the related idea that
there are some true statements that we cannot prove. This discussion
will require that we have some rough notion of proofs and a system
for verifying them.

\begin{defn}[Proofs and Proof Verifiers]
  \label{def:ppv}
  For our purposes we will encode statements $x$ and proofs $w$ as
  binary strings $x, w \in \{0,1\}^*$.

  We will then provide these to a proof verifier $V$. We require $V$
  to satisfy a few conditions:
  \begin{enumerate}
  \item Soundness: $\exists w, V(x, w) = 1 \implies w$ is true. That
    is, there is no proof of a false statement.
  \item Effectiveness: $V$ is computable.
  \end{enumerate}
\end{defn}

Ideally, $V$ would also be \emph{complete}: for every true statement $x'$
there exists some $w'$ that proves it
($V(x', w') = 1$). However, we will see that this is not the
case. There exists a true statement $x$ which does not have a proof
under $V$.

\subsection{Incompleteness from Halting Problem}
We can show that a proof verifier cannot be complete using our
knowledge of the halting problem. We will approach this by reducing
the halting problem to proof verification.

\begin{theorem}
  \label{thm:incomplete}
  A proof verifier $V$ as defined in Definition~\ref{def:ppv} cannot
  be complete.
\end{theorem}
\begin{proof}
  Given a program $P$ and input $x$ we will attempt to prove one of
  the statements ``$P$ halts on input $x$'' and ``$P$ does not halt on
  input $x$'' by brute force. That is, enumerate all possible proofs
  $w \in \{0,1\}^*$ in order of increasing length.

  For each of these, run $V(\text{``$P$ halts on $x$''}, w)$ and
  $V(\text{``$P$ does not halt on $x$''}, w)$. If $V$ accepts one of
  these statements and proofs, we have answered the halting problem on
  $(P, x)$.
\end{proof}

\begin{exercise}
  The proof of Theorem~\ref{thm:incomplete} uses each of the
  properties of soundness, effectiveness and completeness. Where?
\end{exercise}

\ifsolutions
\begin{solution}
  Soundness is used to tell us that we can ``trust'' the answer we get
  from $V$. If $V$ accepts a proof that $P(x)$ does/does not halt, we
  know that that particular statement is true. Effectiveness allows us
  to run the verifier $V$ on our proof strings. Completeness is used
  to assure us that that our procedure of enumerating all the proof
  strings will eventually end and lead $V$ to accept one of the two
  statements. One of the statements that $P(x)$ halts or does not halt
  \emph{must} be true. Therefore if $V$ is complete, then there is
  proof $w$ that $V$ will accept for one of these statements. By
  enumerating all possible $w$ we will eventually find it.
\end{solution}
\else
\vspace{3in}
\fi

\end{document}

%%% Local Variables:
%%% mode: latex
%%% TeX-master: t
%%% TeX-engine: xetex
%%% End:
